\documentclass[12pt]{article}
\usepackage[utf8]{inputenc}
\usepackage[russian]{babel}
\usepackage{amsmath}
\usepackage{hyperref} 
\usepackage{graphicx}
\usepackage{float}
\usepackage{listings}
\usepackage{xcolor}
\usepackage{array}
\usepackage{multirow}

\definecolor{codegreen}{rgb}{0,0.6,0}
\definecolor{codegray}{rgb}{0.5,0.5,0.5}
\definecolor{codepurple}{rgb}{0.58,0,0.82}
\definecolor{backcolour}{rgb}{0.95,0.95,0.92}

\lstdefinestyle{cppstyle}{
    backgroundcolor=\color{backcolour},   
    commentstyle=\color{codegreen},
    keywordstyle=\color{magenta},
    numberstyle=\tiny\color{codegray},
    stringstyle=\color{codepurple},
    basicstyle=\ttfamily\footnotesize,
    breakatwhitespace=false,         
    breaklines=true,                 
    captionpos=b,                    
    keepspaces=true,                 
    numbers=left,                    
    numbersep=5pt,                  
    showspaces=false,                
    showstringspaces=false,
    showtabs=false,                  
    tabsize=2
}

\lstset{style=cppstyle}
\usepackage[left=2cm,right=2cm,top=2cm,bottom=2cm]{geometry}

\begin{document}

\begin{titlepage}
\centering
    \large
    Министерство науки и высшего образования Российской Федерации\\[0.5cm]
    Федеральное государственное автономное образовательное учреждение высшего образования\\[0.5cm]
    \textbf{<<Национальный исследовательский Нижегородский государственный университет им. Н.И. Лобачевского>>}\\
    (ННГУ)\\[1cm]
    Институт информационных технологий, математики и механики\\
\vspace*{\fill}
{\Huge \textbf{Отчет по лабораторной работе}}

\vspace{0.5cm}
{\Huge \textbf{Выделение рёбер на изображении с использованием оператора Собеля}}

\vspace*{\fill}

{\large Нижний Новгород\\ 2024}
\end{titlepage}

\newpage
\section*{Введение}
В данной работе рассматривается задача выделения рёбер на изображении с использованием оператора Собеля. Этот оператор является базовым инструментом для выделения границ на изображениях, так как позволяет определять резкие изменения интенсивности пикселей в горизонтальном и вертикальном направлениях. Для увеличения производительности обработка изображений была распараллелена с использованием MPI.

\newpage
\section*{Постановка задачи}
Необходимо реализовать последовательную и параллельную версии алгоритма выделения рёбер с использованием оператора Собеля. Цель работы — сравнение производительности и корректности обеих версий. Основные этапы работы включают:
\begin{itemize}
    \item Разработка последовательной версии оператора Собеля.
    \item Распараллеливание вычислений с помощью библиотеки MPI.
    \item Проведение экспериментов для оценки ускорения параллельной версии.
\end{itemize}

\newpage
\section*{Описание алгоритма}
Оператор Собеля использует два свёрточных ядра для вычисления градиентов:
\begin{equation}
\text{Sobel Kernel X} = \begin{bmatrix}
-1 & 0 & 1 \\
-2 & 0 & 2 \\
-1 & 0 & 1
\end{bmatrix}, \quad
\text{Sobel Kernel Y} = \begin{bmatrix}
-1 & -2 & -1 \\
 0 &  0 &  0 \\
 1 &  2 &  1
\end{bmatrix}.
\end{equation}

Для каждого пикселя изображения вычисляется градиент:
\begin{equation}
G = \sqrt{G_x^2 + G_y^2},
\end{equation}
где $G_x$ и $G_y$ — результаты свёртки с ядрами X и Y соответственно.

\newpage
\section*{Описание схемы распараллеливания}
Для повышения производительности была реализована параллельная версия алгоритма с использованием MPI. Основная идея распараллеливания:
\begin{itemize}
    \item Изображение делится на части (по строкам) между процессами.
    \item Каждый процесс обрабатывает свою часть изображения.
    \item Результаты объединяются в финальное изображение с помощью MPI.
\end{itemize}
Синхронизация между процессами выполняется с помощью функций \texttt{scatterv} и \texttt{gatherv}.

\newpage
\section*{MPI-версия - описание программной реализации}
Параллельная версия алгоритма состоит из следующих этапов:
\begin{enumerate}
    \item \textbf{Инициализация:} Главный процесс читает изображение, делит его на части и отправляет другим процессам.
    \item \textbf{Обработка:} Каждый процесс выполняет свёртку с ядрами Собеля для своей части изображения.
    \item \textbf{Сбор результатов:} Части обработанного изображения собираются главным процессом.
\end{enumerate}
Пример фрагмента кода распределения данных:
\begin{lstlisting}[language=C++,caption={Распределение данных между процессами}]
boost::mpi::scatterv(world, image, sendcounts, senddispls, locimg.data(), locimg.size(), 0);
\end{lstlisting}

\newpage
\section*{Результаты экспериментов}
Эксперименты проводились на изображении размером $1024 \times 1024$ пикселя. Были измерены следующие параметры:
\begin{itemize}
    \item Время выполнения последовательной версии.
    \item Время выполнения параллельной версии на разном количестве процессов.
    \item Ускорение параллельной версии.
\end{itemize}

Результаты представлены в таблице:
\begin{table}[H]
\begin{tabular}{|l|l|ll|l|}
\hline
\multirow{2}{*}{N}      & \multirow{2}{*}{Seq}       & \multicolumn{2}{l|}{MPI}          & \multirow{2}{*}{$\frac{\text{MPI}}{\text{Seq}}$} \\ \cline{3-4}
                        &                            & \multicolumn{1}{l|}{n} & time     &                           \\ \hline
\multirow{2}{*}{$10^4$} & \multirow{2}{*}{0.005021}  & \multicolumn{1}{l|}{2} & 0.004227 & 0.841864                   \\ \cline{3-5} 
                        &                            & \multicolumn{1}{l|}{4} & 0.003992 & 0.795060                 \\ \hline
\multirow{2}{*}{$10^5$} & \multirow{2}{*}{0.055782}  & \multicolumn{1}{l|}{2} & 0.047529 & 0.852049                 \\ \cline{3-5} 
                        &                            & \multicolumn{1}{l|}{4} & 0.046559 & 0.834660                  \\ \hline
\multirow{2}{*}{$10^6$} & \multirow{2}{*}{0.599623}  & \multicolumn{1}{l|}{2} & 0.498934 & 0,832079                  \\ \cline{3-5} 
                        &                            & \multicolumn{1}{l|}{4} & 0.485892 & 0,810329                  \\ \hline
\end{tabular}
\end{table}


\newpage
\section*{Выводы из результатов}
Параллельная версия алгоритма оператора Собеля показала значительное ускорение при увеличении числа процессов. Однако эффективность параллелизации снижается из-за накладных расходов на передачу данных между процессами. Для больших изображений параллельная версия становится более выгодной.

\newpage
\section*{Заключение}
В работе был реализован алгоритм выделения рёбер на изображении с использованием оператора Собеля. Реализация включала последовательную и параллельную версии. Эксперименты подтвердили корректность работы обеих версий и показали преимущество параллельной реализации для больших изображений.

\newpage
\section*{Литература}
\begin{enumerate}
    \item Gonzalez, R. C., Woods, R. E. \textit{Digital Image Processing}. Pearson, 2018.
    \item Boost MPI Documentation: \url{https://www.boost.org/doc/libs/release/doc/html/mpi.html}
\end{enumerate}

\newpage
\section*{Приложение}
\begin{lstlisting}[language=C++,caption={Заголовочный файл Sobel},label={lst:header}]
#include <array>
#include <boost/mpi/collectives.hpp>
#include <boost/mpi/communicator.hpp>
#include <vector>
// ... код заголовочного файла
\end{lstlisting}
\begin{lstlisting}[language=C++,caption={Реализация Sobel},label={lst:implementation}]
#include "../include/ops_mpi.hpp"
// ... код реализации
\end{lstlisting}

\end{document}
